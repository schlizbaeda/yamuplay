% Meta-Informationen -----------------------------------------------------------
%   Definition von globalen Parametern, die im gesamten Dokument verwendet
%   werden k�nnen (z.B auf dem Deckblatt etc.).
%
%   ACHTUNG: Wenn die Texte Umlaute oder ein Esszet enthalten, muss der folgende
%            Befehl bereits an dieser Stelle aktiviert werden:
%            \usepackage[latin1]{inputenc}
% ------------------------------------------------------------------------------
\newcommand{\titel}{{\Bezeichnung} -- {\BezeichnungLang} f�r den Raspberry Pi}
\newcommand{\untertitel}{}%{TODO: und hier kommt der Untertitel}
\newcommand{\Bezeichnung}{YAMuPlay}
\newcommand{\BezeichnungLang}{Yet Another MUsic Player}
\newcommand{\Version}{V0.2}
\newcommand{\Dokumentart}{D O K U M E N T A T I O N}
\newcommand{\autor}{schlizb�da}

% verwendete Hardware
\newcommand{\RPi}{Raspberry Pi}

% verwendete Software
\newcommand{\omxplayer}{omxplayer.bin}
\newcommand{\github}{GitHub}

%Steuerelemente von Software:
\newcommand{\prompt}[1]{\Code{\textit{#1}}}
\newcommand{\filenam}[1]{\Code{#1}}

\newcommand{\button}[1]{\Code{[{#1}]}}
\newcommand{\menuitem}[1]{\textbf{\textit{"{#1}"}}}
\newcommand{\checkbox}[1]{\textbf{\textit{"{#1}"}}}

\newcommand{\Verein}[1]{\textit{#1}}

%Smileys:
\newcommand{\smiley}[1]{\includegraphics[width=0.3cm]{Bilder/smileys/{#1}}}
