

% Zeilenabstand 1,5 Zeilen -----------------------------------------------------
\onehalfspacing

% ------------------------------------------------------------------------------
% ----------- Seitenr�nder -----------------------------------------------------
% ------------------------------------------------------------------------------

\setlength{\topskip}{\ht\strutbox} % behebt Warnung von geometry
\geometry{paper=a4paper,left=30mm,right=25mm,top=20mm,bottom=40mm}
% zus�tzlich bindingoffset angebbar(linker Rand)

% ------------------------------------------------------------------------------
% ----------- Kopf- und Fu�zeilen ----------------------------------------------
% ------------------------------------------------------------------------------

\pagestyle{scrheadings}
% Kopf- und Fu�zeile auch auf Kapitelanfangsseiten
\renewcommand*{\chapterpagestyle}{scrheadings} 
% Schriftform der Kopfzeile
\renewcommand{\headfont}{\normalfont}

%%% KOPFZEILE %%%
\ihead{\hspace*{16pt} \large{\textsc{\titel}} \\[1ex] \textit{\hspace*{16pt} \headmark}}
\chead{}
%\ohead{\includegraphics[scale=0.06]{\logo}}
\setlength{\headheight}{21mm}           % H�he der Kopfzeile
% Kopfzeile �ber den Text hinaus verbreitern
\setheadwidth[-20pt]{textwithmarginpar} % neg. schiebt nach links, 0 is mittig
\setheadsepline[text]{0.4pt}[\hspace{20pt}]    % Trennlinie unter Kopfzeile [text,head,]

%%% FU�ZEILE %%%
\ifoot{}  %\ifoot{\copyright\ \autor} Autor optional hinzuf�gen
\cfoot{- \pagemark ~-}
\ofoot{}

% ------------------------------------------------------------------------------
% ----------- sonstige typographische Einstellungen ----------------------------
% ------------------------------------------------------------------------------

\frenchspacing % erzeugt ein wenig mehr Platz hinter einem Punkt

% Schusterjungen und Hurenkinder vermeiden
\clubpenalty = 10000
\widowpenalty = 10000 
\displaywidowpenalty = 10000

% Quellcode-Ausgabe formatieren
\lstset{numbers=left, numberstyle=\tiny, numbersep=5pt, breaklines=true}
\lstset{emph={square}, emphstyle=\color{red}, emph={[2]root,base}, emphstyle={[2]\color{blue}}}

% Fu�noten fortlaufend durchnummerieren
\counterwithout{footnote}{chapter}

%\parindent 0pt % kein Einzug nach NewLine